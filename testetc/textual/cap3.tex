% Capitulo 3 - Análise de Sinais
\chapter{\captres}\label{analise}
\par 
Para a proteção dos Sistemas Elétricos de Potência, foram criados vários métodos para diagnóstico de faltas em linhas de transmissão.
Os métodos podem ser divididos em métodos convencionais, métodos baseados em análise de sinais e métodos baseados em sistemas inteligentes.
Os métodos convencionais utilizam uma variedade de parâmetros para tomada de decisão sobre a ocorrência da falta, sendo que os parâmetros mais comuns são a tensão e a corrente da linha.
Os métodos baseados em análise de sinais utilizam os transitórios de alta frequência gerados pela falta na linha para sua detecção, onde os transitórios são extraídos dos sinais de tensão ou corrente, empregando ferramentas matemáticas, tais como a Root Mean Square (RMS), Transformada de Fourier (TF) e Transformada Wavelet (TW).
Os métodos baseados em sistemas inteligentes utilizam dados extraídos da linha como padrões de entrada para o sistema inteligente, tais como Redes Neurais Artificiais (RNAs), Lógica Fuzzy ou Redes Neurofuzzy [INA10]. 
\par
Aborda-se neste capitulo especialmente três métodos que envolvam a análise de sinais assim como conceitos sobre amostragem de dados, filtros e decomposição de sinais.